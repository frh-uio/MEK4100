\documentclass[english,a4paper,12pt]{article}
\usepackage[utf8]{inputenc} %for å bruke æøå
\usepackage{babel}
\usepackage{verbatim} %for å inkludere filer med tegn LaTeX ikke liker
\usepackage[document]{ragged2e}
\bibliographystyle{plain}
\usepackage{amsmath}
\usepackage{ulem}
\usepackage[pdftex]{graphicx}
\usepackage{gensymb}
\usepackage{float}
\usepackage{hyperref}
\usepackage{amssymb}
\usepackage[top=0.6in, bottom=0.8in, left=0.9in, right=0.7in]{geometry}
\usepackage{listings}
\usepackage{color}
\usepackage{tikz}
\usepackage{cancel}
\usepackage{filecontents}

\title{MEK4100\\ Mandatory assignment 1}
\author{Farnaz Rezvany Hesary}
\date{\today}

\begin{document}

\definecolor{codegreen}{rgb}{0,0.6,0}
\definecolor{codegray}{rgb}{0.5,0.5,0.5}
\definecolor{codepurple}{rgb}{0.58,0,0.82}
\definecolor{backcolour}{rgb}{0.95,0.95,0.92}
 
\lstdefinestyle{mystyle}{
    backgroundcolor=\color{backcolour},   
    commentstyle=\color{codegreen},
    keywordstyle=\color{magenta},
    numberstyle=\tiny\color{codegray},
    stringstyle=\color{codepurple},
    basicstyle=\footnotesize,
    breakatwhitespace=false,         
    breaklines=true,                 
    captionpos=b,                    
    keepspaces=true,                 
    numbers=left,                    
    numbersep=5pt,                  
    showspaces=false,                
    showstringspaces=false,
    showtabs=false,                  
    tabsize=2
}
 
\lstset{style=mystyle}

\maketitle

\section*{Problem 3}
\subsection*{a)}
We have the following parameters:\\
\begin{align*}
c&:\bigg(\frac{L}{T}\bigg) \qquad \rho:\bigg(\frac{M}{L^3}\bigg) \qquad g:\bigg(\frac{L}{T^2}\bigg)\\
\sigma&:\bigg(\frac{M}{T^2}\bigg) \hspace{7mm} \lambda:L \hspace{16mm} a:L
\end{align*}
6 parameters - 3 dimensional units = 3 dimensionless numbers.\\

\begin{align*}
\pi_1 = \frac{\sigma}{\rho g \lambda^2} \quad, \quad
\pi_2 =\frac{c^2}{g \lambda} = \frac{c}{\sqrt{g\lambda}} \quad, \quad
\pi_3 = \frac{a}{\lambda}
\end{align*}

\subsection*{b)}
\begin{align*}
\text{when}\quad a \rightarrow 0 \qquad \Rightarrow \qquad \pi_3 \rightarrow 0
\end{align*}
so we have that $G(\pi_1,\pi_2)=0$\\ 
\begin{align*}
\Rightarrow c=\sqrt{g \lambda}f(\pi_1) = \sqrt{g \lambda}f(\frac{\sigma}{\rho g \lambda^2})
\end{align*}
where $\pi_1$ expresses the importance of surface tension in relation to gravity, concerning the wave celerity.

\newpage

\section*{Problem 12}
We have the following equation:\\
\begin{align*}
x^2 - 2x + 1 = \epsilon (1 + 2x)
\end{align*}
\subsection*{a)}
For this part of the exercise we are going to try the straightforward, naive, perturbation scheme.\\
We assume $x=x_0 + \epsilon x_1 + \epsilon^2 x_2 + ...$\\

\begin{align*}
&\Rightarrow (x_0 + \epsilon x_1 + \epsilon^2 x_2)^2 -2(x_0 + \epsilon x_1 + \epsilon^2 x_2) + 1 -\epsilon -2\epsilon(x_0 + \epsilon x_1 + \epsilon^2 x_2)=0\\\\
&\Rightarrow x_0^2 + x_0 x_1 \epsilon + x_0 x_2 \epsilon^2 + x_0 x_1 \epsilon + x_1^2 \epsilon^2 + x_1 x_2 \epsilon^3 + x_0 x_2 \epsilon^2 + x_1 x_2 \epsilon^3 + x_2^2 \epsilon^4\\ 
&-2x_0 -2x_1 \epsilon -2 x_2 \epsilon^2 + 1 -\epsilon-2 \epsilon x_0 - 2 \epsilon^2 x_1 -2 x_2 \epsilon^3 = 0\\\\
&\Rightarrow x_0^2 -2x_0 +1 + \epsilon(2x_0 x_1 -2x_1 -1 -2x_0) + \epsilon^2(2x_0x_2 + x_1^2 -2x_1 -2x_2)=0
\end{align*}

We can now collect the terms for $\epsilon^0$ and $\epsilon^1$.\\

\begin{align*}
\epsilon^0 &: x_0^2 -2x_0 +1 = 0 \quad \Rightarrow \quad x_0 = 1\\\\
\epsilon^1 &: 2x_0 x_1 - 2x_1 = 1 + 2x_0 \quad (\text{ here we can set $x_0 =1$})\\
&\Rightarrow \quad 2x_1 - 2 x_1 = 3
\end{align*}

This gives us that $0 = 3$ which clearly means that we a get a break down of the direct perturbation technique.\\
So $x = x_0 + \epsilon x_1$ must be a invalid form for the solution.

\subsection*{b)}
In this part of the exercise, we are going to employ a general tye expansion $x = x_0 + x_1 + x_2 + ...$,where $x_0 \gg x_1 \gg x_2 ...$, when $\epsilon \rightarrow 0^+$\\
We try at first a solution of the form: $x=x_0 + x_1$ where $x_0 = 1$\\
\begin{align*}
\Rightarrow \quad (x_0 + x_1)^2 -2(x_0 + x_1) +1 &= \epsilon(1 + 2x_0 + 2x_1)\\
1 + 2x_1 + x_1^2 -2 -2x_1 + 1 &= \epsilon(3 + 2x_1)\\
\Rightarrow x_1^2 &= 3\epsilon + 2\epsilon x_1
\end{align*}

\begin{align*}
&x_1^2 -2\epsilon x_1 - 3\epsilon = 0\\
&\textcircled{1} \hspace{7mm} \textcircled{2} \hspace{7mm} \textcircled{3}
\end{align*}
Now we can use the method of dominant balancing on these 3 terms to find $x_1$:\\

\begin{align*}
\textcircled{1} \sim \textcircled{2}  &\Rightarrow x_1^2 - 2 \epsilon x_1 = 0 \quad \Rightarrow x_1 - 2\epsilon = 0 \quad \Rightarrow x_1 = 2\epsilon\\
&\Rightarrow \textcircled{3} \gg \textcircled{1} \text{ and} \textcircled{2} \qquad \text{NO!}\\\\
\textcircled{1} \sim \textcircled{3}  &\Rightarrow x_1^2 - 3\epsilon=0 \quad \Rightarrow x_1 = \pm \sqrt{3 \epsilon} \qquad \text{OK!}\\\\
\textcircled{2} \sim \textcircled{3}  &\Rightarrow -2\epsilon x_1 - 3 \epsilon = 0 \quad \Rightarrow x_1 = \frac{-3}{2} \quad \text{which gives us } \quad \frac{9}{4} = 0 \quad \text{so NO!}
\end{align*}
Then we have that $x_1 = \pm \sqrt{3\epsilon}$ and $x_0 = 1$.
In order to find an expression for $x_2$ we need to try with a solution of the form $x=x_0 + x_1 + x_2$, which gives us the following expression:\\
$$(x_0 + x_1 + x_2)^2 -2(x_0 + x_1 + x_2) +1 = \epsilon(1+2(x_0 + x_1 + x_2))$$

\begin{align*}
\Rightarrow \quad &x_0^2 + x_0 x_1 + x_0 x_2 + x_0 x_1 + x_1^2 + x_1 x_2 + x_2 x_0 + x_2 x_1 + x_2^2 -2x_0 -2x_1 -2x_2 +1\\ 
&= \epsilon(1+2x_0 + 2x_1 +2x_2)
\end{align*}
We can now insert the values we found for $x_0$ and $x_1$ into the expression above. This will give us:\\
\begin{align*}
&1 \pm \sqrt{3 \epsilon} + x_2 \pm \sqrt{3 \epsilon} + 3 \epsilon \pm \sqrt{3 \epsilon}x_2 + x_2 \pm \sqrt{3 \epsilon}x_2 + x_2^2 -2 \pm 2 \sqrt{3 \epsilon} -2x_2 + 1\\ 
&= \epsilon(3 \pm 2\sqrt{3 \epsilon} + 2x_2)\\\\
\Rightarrow &x_2^2 \pm 2x_2 \sqrt{3 \epsilon} - 2\epsilon x_2 \pm 2 \epsilon \sqrt{3 \epsilon} = 0\\
&\textcircled{1} \hspace{9mm} \textcircled{2} \hspace{12mm} \textcircled{3} \hspace{10mm} \textcircled{4}
\end{align*}
Again we can use the method of dominant balancing on these 4 terms to find $x_2$\\
\begin{align*}
&\textcircled{1} \sim \textcircled{2} \quad \Rightarrow \quad x_2^2 \pm 2x_2\sqrt{3 \epsilon}=0 \quad \Rightarrow \quad x_2 = \pm 2\sqrt{3\epsilon}\\
&\text{This gives us $x_2 > x_1$, but we know that $x_0 \gg x_1 \gg x_2 $ so NO!}\\\\
&\textcircled{1} \sim \textcircled{3} \quad \Rightarrow \quad x_2^2 - 2 \epsilon x_2 =0 \quad \Rightarrow \quad x_2 = 2\epsilon\\
&\Rightarrow \textcircled{1} \sim \epsilon^2 \quad \textcircled{2} \sim \epsilon^{\frac{3}{2}}  \quad  \textcircled{3} \sim \epsilon^2 \quad  \textcircled{4} \sim \epsilon^{\frac{3}{2}}\\
&\text{we can see that \textcircled{1} \text{ and} \textcircled{3} $<$ \textcircled{2} \text{ and} \textcircled{4} so NO!}\\\\
&\textcircled{1} \sim \textcircled{4} \quad \Rightarrow \quad x_2^2 \pm 2\epsilon\sqrt{3\epsilon}=0 \quad \Rightarrow \quad x_2 = \pm\sqrt{2\epsilon}\sqrt[4]{3\epsilon} \sim \epsilon^{\frac{3}{4}}\\
&\Rightarrow \textcircled{1} \sim \epsilon^{\frac{6}{4}} \quad \textcircled{2} \sim \epsilon^{\frac{5}{4}}  \quad  \textcircled{3} \sim \epsilon^{\frac{7}{4}} \quad  \textcircled{4} \sim \epsilon^{\frac{3}{4}}\\
&\text{we can see that \textcircled{1} $<$ \textcircled{2} \quad so NO!}\\\\
&\textcircled{2} \sim \textcircled{3} \quad \Rightarrow \quad \pm 2 x_2 \sqrt{3 \epsilon} - 2 \epsilon x_2 = 0 \quad \Rightarrow \quad x_2 = 0 \quad \text{NO!}\\\\
&\textcircled{2} \sim \textcircled{4} \quad \Rightarrow \quad 2 x_2 \sqrt{3 \epsilon} \pm 2 \epsilon \sqrt{3 \epsilon}= 0 \quad \Rightarrow \quad x_2 = \epsilon\\
&\Rightarrow \textcircled{1} \sim \epsilon^2 \quad \textcircled{2} \sim \epsilon^{\frac{3}{2}}  \quad  \textcircled{3} \sim \epsilon^2 \quad  \textcircled{4} \sim \epsilon^{\frac{3}{2}} \quad \text{OK!}\\
\end{align*}
We have now found that $$x = x_1 + x_1 + x_3 = 1 \pm \sqrt{3 \epsilon} + \epsilon $$
\subsection*{c)}
In this part of the exercise we are going to solve $(x-1)^n=\epsilon x$. We can start by solving the unpertubed problem to find $x_0$.\\

\begin{align*}
(x_0 - 1)^n = 0 \quad \Rightarrow \quad x_0 = 1 \quad \text{when n is a positive integer}
\end{align*}

Now we can try with a solution of the form $x=x_0 + x_1$ to find $x_1$

\begin{align*}
((x_0 + x_1) - 1)^n &= \epsilon(x_0 + x_1)\\
((1 + x_1) - 1)^n &= \epsilon + \epsilon x_1\\\\
x_1^n - \epsilon x_1 - \epsilon &= 0
\end{align*}
\hspace{65mm}\textcircled{1} \hspace{3mm} \textcircled{2} \hspace{3mm} \textcircled{3}\\
Dominant balance method:\\

\begin{align*}
&\textcircled{1} \sim \textcircled{2} \quad \Rightarrow \quad x_1^n - \epsilon x_1 = 0 \quad \Rightarrow \quad x_1^{n-1} = \epsilon \quad \Rightarrow \quad x_1 = \epsilon^{\frac{1}{n-1}}\\
&\Rightarrow \textcircled{1} \sim \epsilon^{\frac{n}{n-1}} \quad \textcircled{2} \sim \epsilon^{\frac{n}{n-1}}  \quad  \textcircled{3} \sim \epsilon\\
&n \neq 1 \quad \Rightarrow \quad n \geq 2 \quad \Rightarrow \quad \textcircled{1} \text{ and} \textcircled{2} \ll \textcircled{3} \quad \text{NO!}\\\\
&\textcircled{1} \sim \textcircled{3} \quad \Rightarrow \quad x_1^n - \epsilon = 0 \quad \Rightarrow \quad x_1 = \epsilon^{\frac{1}{n}}\\
&\Rightarrow \textcircled{1} \sim \epsilon \quad \textcircled{2} \sim \epsilon^{\frac{n+1}{n}}  \quad  \textcircled{3} \sim \epsilon\\
&\textcircled{1} \text{ and } \textcircled{3} \gg \textcircled{2} \quad \text{OK!}
\end{align*}

This gives us:\\
$$x = 1 + \epsilon^{\frac{1}{n}}$$

\newpage 

\section*{Problem 14}
We have the following first order, non separable differential equation:\\
\begin{align*}
y' + y + \epsilon x y^2 &= 0\\
y(0) &= 1
\end{align*}
where $\epsilon$ is very small.\\ 
We can begin with solving the unpertubed problem to find $y_0$.\\
\begin{align*}
y' + y &= 0 \qquad \text{This equation is separable}\\
\frac{y'}{y} &= -1 \quad \Rightarrow \quad ln(y) = -x + C \quad \Rightarrow \quad y= e^{-x + C}=Ce^{-x}\\
y(0) &= 1 \qquad \Rightarrow \quad y(0) = Ce^0 = 1 \quad \Rightarrow \quad C=1
\end{align*}
\begin{align*}
\Rightarrow y_0 = e^{-x}
\end{align*}
In order to find $y_1$, we can try with a solution of the form: $y = y_0 + \epsilon y_1$
\begin{align*}
&(y_0 + \epsilon y_1)' + (y_0 + \epsilon y_1) + \epsilon x (y_0 + \epsilon y_1)^2 = 0\\
\end{align*}
The initial condition then becomes:\\

\begin{align*}
y_0(0) + \epsilon y_1(0) = 1 \quad \text{where:} \quad y_0(0) = 1 \quad \text{and} \quad y_1(0) = 0
\end{align*}

\begin{align*}
y_0' + \epsilon y_1' + y_0 + \epsilon y_1 + \epsilon x y_0^2 + 2\epsilon^2 x y_0 y_1 + \epsilon^3 x y_1^2 = 0\\
\end{align*}

We can then collect all the terms for $\epsilon^0$, $\epsilon^1$, $\epsilon^2$ and $\epsilon^3$:
\begin{align*}
&\epsilon^0 : y_0' + y_0 = 0\\
&\epsilon^1 : y_1' + y_1 + xy_0^2 = 0 \quad \Rightarrow \quad y_1' + y_1 + xe^{-2x} = 0\\
&\epsilon^2 : 2x y_0 y_1 = 0\\
&\epsilon^3 : xy_1^2 = 0
\end{align*}
We can see that we need to solve $y_1' + y_1 + xe^{-2x} = 0$ with initial condition $y_1(0) = 0$ to find $y_1$.\\
The integrating factor is $\mu(x) = e^{\int 1 dx} = e^x$.
\begin{align*}
e^x y_1' + e^x y_1 &= e^x (-xe^{-2x}) = -e^{-x}x\\
\frac{d}{dx}(e^x y_1(x)) &= -e^{-x}x\\
\Rightarrow \quad e^x y &= \int -e^{-x}x dx = - \int e^{-x}xdx
\end{align*}
We can then integrate by parts:\\
\begin{align*}
\Rightarrow \quad -\int e^{-x} dx = e^{-x}x + \int e^{-x} dx
\end{align*}
We can then use substitution in the integral: $\quad u = -x \quad du = -dx$
\begin{align*}
\Rightarrow \quad e^{-x}x + \int e^u &= xe^{-x} + e^u + C = xe^{-x} + e^{-x} + C = e^{-x} (x+1) + C\\
e^x y_1 &= e^{-x}(x+1) + C = xe^{-x} + e^{-x} + C\\
y &= x e^{-2x} + e^{-2x} + Ce^{-x} = e^{-2x}(x+1+Ce^x)
\end{align*}
The initial condition gives:\\
\begin{align*}
y_1(0) = e^0 + C e^0 = 0 \quad \Rightarrow \quad 1+C = 0 \quad \Rightarrow \quad C=-1
\end{align*}
We will then have:\\

\begin{align*}
y_1 &= e^{-2x}(x - e^x +1) = x e^{-2x} + e^{-2x} - e^{-x}\\\\
\Rightarrow \quad y &= y_0 + \epsilon y_1 = e^{-x} + x e^{-2x} + e^{-2x} - e^{-x} = e^{-x} + \epsilon((x+1)e^{-2x} - e^{-x})
\end{align*}
\end{document}